\chapter{Consumer Utility}
So far, we have dealt primarily with firms and how they decide what to produce. Now, we address the other side of the market: consumers and how they decide what to buy and consume. However, we need to find a way to convert this to a maximization problem. With firms, we made the very reasonable assumption that they would try to maximize profits. However, with individuals it is less clear what they would be maximizing. In economics, we assume that individuals are maximizing a \vocab{utility function}, which, in a somewhat tautological definition, is simply whatever an individual maximizes when they are making choices. 

\section{What is utility?}
You may have seen utility in previous economics courses described as a quantification of the ``happiness'' of individuals, and the utility function describes how many ``utils'' that an individual receives from consuming certain goods. This may be a useful way of thinking about the utility function and can add some valuable insights, but we want a more formal treatment of utility functions that does not rely on something as abstract and non-specific as representing ``happiness.'' However, this leaves us with a series of problems. Can this happiness be measured and observed? Is it the same across people? Can different types of happiness be compared? In this section, we will explore the formal treatment of utility in economics that defines utility in a way that handles some of these issues and avoids others. It will not be important for you to understand every aspect in this approach, and we will avoid delving into the fully formal technicalities of utility, but it may be useful for you understand where utility comes from to know what you can and cannot do with utility functions.

\subsection*{Utility as preference relations}
We start by approaching the problem of quantifying an individual's preferences by considering a simpler problem: determining whether an individual prefers one outcome over another. Suppose we have two outcomes, $A$ and $B$, for a given individual. These could be any set of outcomes. $A$ might represent receiving 4 apples and $B$ might represent receiving 6 bananas, or $A$ might represent going to Harvard while $B$ is going to Yale.

We have a fairly reasonable to judge whether an individual prefers outcome $A$ or outcome $B$, by observing which they choose when presented with a choice. This means that we can denote a \vocab{preference relation} on outcomes, which expresses which outcome an individual prefers between two outcomes. The notation is as follows:
\begin{itemize}
    \item $A \prec B$ means that the agent strictly prefers $B$ to $A$. That is, given the choice between $A$ and $B$, the agent would choose $B$.
    \item $A \succ B$ means that the agent strictly prefers $A$ to $B$.
    \item $A \sim B$ means that the agent is indifferent between $A$ and $B$.
    \item $A \precsim B$ means that the agent weakly prefers $B$ to $A$. That is, either $A \prec B$ or $A \sim B$.
    \item $A \succsim B$ means that the agent weakly prefers $A$ to $B$.
\end{itemize}
This allows us to rigorously define an individual's preferences by a preference relation on the possible outcomes. However, this alone leaves us with a bit too much room. In order to have useful preferences, we need to assume that agents have \vocab{rational preference}. However, what economists mean by rational preferences is not a normative description of rationality. An economist makes no judgement, for example, on whether prefering chocolate to vanilla ice cream is ``rational.'' Instead, we define rationality by the following two axioms:
\begin{description}
    \item[Completeness] For any two outcomes $A$ and $B$, exactly one of the following holds: $A \prec B$, $A \succ B$, or $A \sim B$. This axiom tells us two things. The first is that the agent always has some preference between any two outcomes, even if that preference is to be indifferent. The second is that an agent cannot simultaneously prefer $A$ to $B$ and prefer $B$ to $A$.
    \item[Transitivity] For any outcomes $A$, $B$, and $C$, $A \precsim B$ and $B \precsim C$ imply that $A \precsim C$. That is, if we prefer $B$ to $A$ and $C$ to $B$, then we must prefer $C$ to $A$ as well.
\end{description}
Whether you think that these axioms are required to be considered rational is up to you, but for the purposes of microeconomics, we impose these requirements and assume our agents to have rational preferences.

\subsection*{From preference relations to utility functions}
Now that we can define an individual's preferences, we can try to convert them into a utility function. A utility function is simply a way of expressing these preference relationships over outcomes by mapping each outcome to a real number, and outcomes that are more preferred have a higher value. Formally:
\begin{definition*}[Utility function]
    A function $u: X \to R$ is a utility function for a preference relation $\precsim$ if for $A, B \in X$, $A \precsim B \iff u(A) \leq u(B)$. 
\end{definition*}
Notice that the utility function for a given set of preferences is not unique. To make the idea of a utilty function more concrete, let's consider a simple example with a finite set of outcomes.
\begin{example*}
    Let $X = \{A, B, C\}$ be the set of outcomes. Maybe $A$ is getting an apple, $B$ is getting a banana, and $C$ is getting a coconut. Suppose we have a preference relation $\precsim$ where $A \precsim B \precsim C$. We want to construct a utility function $u$ that expresses this preference relation. We might define $u$ as follows:
    \begin{align*}
        u(A) = 1, u(B) = 2, u(C) = 3
    \end{align*}
    Notice that because $C$ is preferred to $B$, $u(C)$ is greater than $u(B)$, and the same is true for all pairs of preference relations. However, this is not the unique representation of the preference relations. Define $\tilde{u}$ as the same as $u$ except with the output doubled:
    \begin{align*}
        \tilde{u}(A) = 2, \tilde{u}(B) = 4, \tilde{u}(C) = 6
    \end{align*}
    Notice that this still represents the preference relation $\precsim$, but has different values than $u$ does. 
\end{example*}
The above example illustrates an important point. Utility functions are ordinal, not cardinal. That is, the magnitude of the difference between $u(A)$ and $u(B)$ does not matter, but the sign does. We can state this more formally:

\begin{proposition*}
    Let $u : X \to \R$ be a utility function representing a preference relation $\precsim$. Let $f : \R \to \R$ be a monotonically increasing function. Then $f \circ u: X \to \R$ is also a utility function representing $\precsim$. 
\end{proposition*}
\begin{proof}
    Let $A, B \in X$ where $A \precsim B$. Then $u(A) \leq u(B)$. By monotonicity of $f$, we also have that $f(u(A)) \leq f(u(B))$. Since $A$ and $B$ were arbitrary, this holds for all $A \precsim B$. So, $f \circ u$ is a utility function for $\precsim$. 
\end{proof}
This tells us that we can add, multiply, apply a positive exponent, take logarithms, or apply any monotonic function to a utility function and keep the same underlying preferences. 

However, this also tells us that you \textbf{cannot compare utilities across individuals}. That is, we can not decide that one person is happier than another because they receive more utility, nor can we say that maximizing utility is in general a desirable goal. Those are cases of normative utility functions, but in our case we only deal with the formally defined utility function. Throughout this text and in the course, we may say that higher utility corresponds to an agent being ``happier,'' but this is merely shorthand and to acheive intuition, and should not be interpreted as a claim on utility actually mapping to happiness. 

There is one last wrinkle in our construction of the utility function. In the finite case, or even in the countably infinite case, the above rationality axioms are sufficient to construct a utility function from a preference relation. However, we might have cases where the set of outcomes is uncountably infinitely large. For example, if you have a utility function over how much money you receive, in which case the outcome space is all real numbers. The rationality axioms alone are insufficient to guarantee the existence of a well-defined utilty function for a preference relation over uncountably infinite axioms in this case. So, we need an additional axiom.
\begin{description}
    \item[Contiunity of preferences] For any sequence of outcome pairs, $\{(x^n, y^n)\}_{n = 1}^\infty$ where $x^n \succsim y^n$ for all $n$, and $x = \lim_{n \to \infty} x^n, y = \lim_{n \to \infty} y^n$, then $x \succsim y$. 
\end{description}
The above is a bit more mathematically formal than required in this course, and you do not need to know the continuity property. It basically says that our preference relations are preserved under limits. However, the key is that if $\precsim$ is a continuous preference relation, then we have a \emph{continuous} utility function $u: X \to \R$ representing $\precsim$. 

While this guarantees that there is a continuous utility function, it does not say anything about the differentiability or other properties of the utility function. However, now that we have established the formal mathematical foundations of utility, we can impose more structure to handle the consumer problem specifically. We will do so in the following section.

\section{The consumer's problem}
The consumer's problem is in some sense the foundation of all economics. It has to do with individuals trying to achieve the best outcome that they can. That is, they are maximizing utility. In this section, we describe the basic setup of the model, the assumptions in the model, and some basic properties from solving the model.

\subsection*{Model setup}
We consider a set of $n$ goods that a consumer can consume, and that the consumer chooses real quantities of each good. We denote the choice for amount of these goods $\vec{x} = (x_1, \dots, x_n)$. This means that our space of ``outcomes'' is $X = \R^n$. We assume our agent has a continuously differentiable utility function $u: \R^n \to \R$. We make a few additional asssumptions on the utility function.
\begin{description}
    \item[Increasing in goods] We assume that $u$ is increasing in each good. Mathematically, this is $\partials{u}{x_i} > 0$ for all $i$. A key assumption here is that the consumer wants each good, and that there are no ``bads.'' There will be cases where this assumption no longer holds for a general utility maximization problem (pollution or garbage for example), but in this case we assume the agent can only be happier with there allocation. This also assumes non-satiation, so that agents always want more of the good. 
    \item[Concavity] We assume that $u(\vec{x})$ is concave in $\vec{x}$. Since $u$ is differentiable, this tells us that $\frac{\partial^2 u}{\partial x_i^2} < 0$ for all $x_i$. The intuition here is that agents tend to have diminishing marginal returns. The 10th chocolate bar adds less additional happiness than the first chocolate bar does.
\end{description}

However, there is a slight problem here, which is that clearly the optimal action for an agent given these assumptions is just to consume an infinite amount of everything. In the real world this does not occur because we have a limited amount of money. So we assume that agents have an exogenous fixed income $y$ that can be spent on purchasing goods. Each good $i$ also has a positive price $p_i > 0$, which we assume the agent takes as exogenous, yielding the price vector $\vec{p}$. This implicitly assumes that the agent is a price-taker, so that the amount of good that the agent purchases has no effect on the price, which is the case if the agent is a relatively small spender in an economy with many other consumers and firms to buy from. This gives us the agent's \vocab{budget constraint},
\begin{align*}
    \vec{p} \cdot \vec{x} = \sum_{i = 1}^n p_i x_i \leq y
\end{align*}
So, the consumer's problem can be summarized as follows:
\begin{align*}
    \max_{\vec{x} \in \R^n} u(\vec{x}) \text{ s.t. } \vec{p} \cdot \vec{x} \leq y
\end{align*}
