\chapter{Slutsky Equation}

In addition to measuring price changes, we might also be interested in understanding the effect of price changes on consumption. Intuitively, if the price of a single good increases, we can separate the effect on consumption into two channels. First, that good becomes relatively more expensive and other goods become relatively cheaper; this effect is called the \vocab{substitution effect}. Second, the consumer has effectively become poorer in terms of the options she can afford; this effect is called the \vocab{income effect}. This section will study the \vocab{Slutsky equation}, which uses our ideas of Marshallian and Hicksian demand to derive an equation that contains both of these pieces together.

\section{Derivation}
Recall from duality\TODO{Add reference} that at an optimum, the Marshallian and the Hicksian must be equal. Focusing on a single good $i$, this implies that, 
$$x_{i}(\mathbf{p}, e(\mathbf{p}, \bar{u}))=h_{i}(\mathbf{p}, \bar{u}),$$

By the Implicit Function Theorem, we can differentiate both sides of this equation with respect to $p_i$, yielding 
$$\frac{d}{d p_{i}} x_{i}(\mathbf{p}, e(\mathbf{p}, \bar{u}))=\frac{d}{d p_{i}} h_{i}(\mathbf{p}, \bar{u}).$$
Using the multivariate chain rule, this becomes 
$$\frac{\partial x_{i}}{\partial p_{i}}(\mathbf{p}, e(\mathbf{p}, \bar{u}))+\frac{\partial x_{i}}{\partial y}(\mathbf{p}, e(\mathbf{p}, \bar{u})) \frac{\partial e}{\partial p_{i}}(\mathbf{p}, \bar{u})=\frac{\partial h_{i}}{\partial p_{i}}(\mathbf{p}, \bar{u}).$$
By Shephard's Lemma, we know $\pdv{e}{p_i}(\mathbf{p}, \bar{u})=h_{i}(\mathbf{p}, \bar{u}),$ so we can simplify our earlier equation to 
$$\frac{\partial x_{i}}{\partial p_{i}}(\mathbf{p}, e(\mathbf{p}, \bar{u}))+h_{i}(\mathbf{p}, \bar{u}) \frac{\partial x_{i}}{\partial y}(\mathbf{p}, e(\mathbf{p}, \bar{u}))=\frac{\partial h_{i}}{\partial p_{i}}(\mathbf{p}, \bar{u}).$$

We now use the first fact that the expenditure function and the indirect utility function are inverses to rewrite this equation in terms of some budget $y$ instead of $\bar{u}$. Define $y = e(\vec{p}, \bar{u})$. Then $\bar{u} = v(p, y)$ by the first fact, and $h_i(\vec{p}, \bar{u}) = x_i(\vec{p}, y)$ by the second fact. We can substitute these terms into our equation to get 
$$\frac{\partial x_{i}}{\partial p_{i}}(\mathbf{p}, y)+x_{i}(\mathbf{p}, y) \frac{\partial x_{i}}{\partial y}(\mathbf{p}, y)=\frac{\partial h_{i}}{\partial p_{i}}(\mathbf{p}, v(\mathbf{p}, y)).$$
Rearranging these terms gives us the famous \vocab{Slutsky equation}
$$
\boxed{\frac{\partial x_{i}}{\partial p_{i}}(\mathbf{p}, y)=\frac{\partial h_{i}}{\partial p_{i}}(\mathbf{p}, v(\mathbf{p}, y))-x_{i}(\mathbf{p}, y) \frac{\partial x_{i}}{\partial y}(\mathbf{p}, y).}
$$
Suppressing the arguments, we can write this more succinctly as 
$$\frac{\partial x_{i}}{\partial p_{i}}=\frac{\partial h_{i}}{\partial p_{i}}-x_{i} \frac{\partial x_{i}}{\partial y}.$$

\section{Interpretation}
The effect of price changes on consumption is exactly the $\frac{\partial x_{i}}{\partial p_{i}}$ term on the left, while the right hand side of the equation separates the substitution and income effects we were interested in. 

\begin{description}

\item[Substitution effect.] This component is given by the first term
$$\frac{\partial h_{i}}{\partial p_{i}}(\mathbf{p}, v(\mathbf{p}, y)).$$
Intuitively, this term represents the first-order change solely from good $i$ becoming more expensive relative to other goods, assuming that we could still afford the original consumption bundle. Another explanation is that this term represents the effect of a price change if real income is kept constant, where real income is defined in terms of the ideal price index discussion from the previous section.

We claim that the substitution effect is always negative; intuitively, if an item becomes relatively more expensive, the substitution effect would cause a shift in spending to the now relatively cheaper items. This fact is called the \vocab{Law of Compensated Demand.}
\begin{proof}
Recall that the expenditure function is concave in $\vec{p}$, so 
$$
\frac{\partial^{2} e}{\partial p_{i}^{2}}(\mathbf{p}, \bar{u})<0.
$$
By Shephard's Lemma, this means
$$
\frac{\partial}{\partial p_{i}} \frac{\partial e}{\partial p_{i}}(\mathbf{p}, \bar{u})=\frac{\partial h_{i}}{\partial p_{i}}(\mathbf{p}, \bar{u})<0.
$$
\end{proof}

\item[Income effect.] This component is given by the second term
$$-x_{i} \frac{\partial x_{i}}{\partial y}.$$
Intuitively, if the consumer becomes $\Delta y$ poorer, the consumption of good $i$ decreases by 
$$
\Delta x_{i} \approx \frac{\partial x_{i}}{\partial y} \Delta y.
$$
If the consumer was originally buying $x_i$ of good $i$, then a price increase of $\Delta p_i$ effectively makes her poorer by
$$
\Delta y \approx -x_{i} \Delta p_{i}.
$$
Putting these pieces together, the income effect of a small price change $\Delta p_i$ is 
$$
\Delta x_{i} \simeq-x_{i} \frac{\partial x_{i}}{\partial y} \Delta p_{i},
$$
which matches our result from Slutsky's equation.

What is the sign of the income effect? It depends! If the income effect is negative (i.e. if $\frac{\partial x_{i}}{\partial y}>0$) then $\frac{\partial x_{i}}{\partial p_{i}}<0$, since the substitution effect is always negative. This means that the demand function is downward sloping, as we might imagine intuitively. However, this might not always be the case; we might have a positive income effect when $\frac{\partial x_{i}}{\partial y}<0$ for some items (e.g. cheap dorm food) that we consume less when we become richer. 

\end{description} 

\section{Income and Price Elasticity}

In order to better break down why the income effect can be positive, we start by differentiating our original budget constraint
$$y = \vec{p} \cdot \vec{x}$$
in order to identify a few important elements. Differentiating both sides with respect to $y$ yields
$$1=\sum_{i=1}^{n} p_{i} \frac{\partial x_{i}}{\partial y}.$$
Manipulating terms, we can write this as 
$$
\sum_{i=1}^{n} \frac{p_{i} x_{i}}{y} \frac{y \partial x_{i}}{x_{i} \partial y}.
$$

We now define two new terms. We define 
$$\alpha_i \equiv \frac{p_{i} x_{i}}{y}$$
to be the \vocab{budget share} of good $i$, which represents the fraction of our budget that we spend on good $i$. We define 
$$\varepsilon_y^i \equiv \frac{y \partial x_{i}}{x_{i} \partial y}$$
to be the \vocab{income elasticity of demand} for good $i$. Intuitively, this tells us the percentage change in demand for a small percentage change in income. You may have seen the elasticity of income in previous economic classes defined as,
\begin{align*}
    \varepsilon_y^i = \frac{\% \text{ change in demand}}{\% \text{ change in income}}
\end{align*}

How can we square these two definitions? We can start by rewriting our definition of the elasticity to be
\begin{align*}
    \varepsilon_y^i \equiv \frac{\partial x_i}{x_i}  \frac{y}{\partial y} = \frac{\partial x_i / x_i}{\partial y / y}
\end{align*}
We abuse the definition of the derivative slightly and treat it as a fraction even though this is not the case formally. Notice however that $\frac{\partial x_i}{x_i}$ is a small change in $x_i$ divided by current consumption of $x_i$, which is a small percentage change in $x_i$. Following the same logic, $\frac{\partial y}{y}$ is a small percentage change in income. So, our definition of income elasticity is in fact equivalent to the interpretatation as small percentage changes.

Another sometimes useful fact is that the income elasticity is definable in terms of log derivatives:
\begin{align*}
    \varepsilon_y^i = \frac{\partial x_i y}{x_i \partial y} = \frac{\partial \log x_i }{\partial \log y_i}
\end{align*}
In general, you can think of a change in the value of the logarithm of a value as a percentage change of said value. 

Notice the relationship between income elasticity and the income effect for a good: the income effect for a good $i$ is negative if and only if $\varepsilon_y^i > 0$, which makes sense given our intuitive understanding of elasticity.

Substituting these new terms into our previous equation, we have the relationship
$$
\sum_{i=1}^{n} \alpha_{i} \varepsilon_{y}^{i} = 1.
$$
The term on the left is a weighted average of the different $\varepsilon_{y}^{i}$ terms, since the some of all our $\alpha_i$ terms must be 1. Since the term on the left is in units of elasticity, we can almost think of this relationship as the idea that with a $1\%$ increase in budget, we can buy $1\%$ more ``stuff'' (not necessarily $1\%$ more quantity of consumption), where the composition of this ``stuff'' is determined by each good's budget share and income elasticity. \TODO{I'm not sure if this explanation makes any sense; I can remove it if it doesn't.} Notice that this relationship also tells us that it is impossible for every good in our option set to have a negative income elasticity; if our budget increases, we should not be buying less of everything. Later, we will see that this means that it is impossible for every good in our option set to be an inferior good.

We can derive a similar relationship to help us understand the substitution effect if we instead differentiate our original budget constraint with respect to the price of some good $p_j$. From the product rule, we get
$$
0=x_{j}+\sum_{i=1}^{n} p_{i} \frac{\partial x_{i}}{\partial p_{j}}.
$$
Multiplying by $\frac{p_j}{y}$ and rearranging yields 
$$
-\frac{p_{j} x_{j}}{y}=\sum_{i=1}^{n} \frac{p_{i} x_{i}}{y} \frac{p_{j} \partial x_{i}}{x_{i} \partial p_{j}}.
$$

We define 
$$\varepsilon_{p_j}^i \equiv \frac{p_j \partial x_{i}}{x_{i} \partial p_j}$$
as the \vocab{cross-price elasticity of demand}, which represents how sensitive our demand for good $i$ is in response to changes in the price of some other good $j$. When $i=j$, then this term is simply the \vocab{price elasticity of demand}. If $\varepsilon_{p_j}^i = 1$ always, then every $1\%$ increase in the price of good $j$ results in a $1\%$ increase in the consumption of good $i$.

Substituting our definitions for cross-price elasticity and budget share, we get the relationship
$$
-\alpha_{j}=\sum_{i=1}^{n} \alpha_{i} \varepsilon_{p_{j}}^{i}.
$$
This equation tells us that the weighted average of cross-price elasticities of all goods with respect to price changes for good $j$ is equal the negative budget share of good $j$. We will more closely examine the $\frac{\partial x_{i}}{\partial p_j}$ piece of our definition of cross-price elasticity later in the section on the general Slutsky equation.

\section{Types of Goods}

We might already have some intuitive idea of what goods have a negative income effect (i.e. we consume more when we get richer) and what goods have a positive income effect (i.e.e we consume less when we get richer. We can further categorize these types of goods using our definition of income elasticity:
\begin{description}
\item[\vocab{Inferior goods}] have a negative income elasticity $\varepsilon_y^i < 0$, so their income effect is positive. That is, we buy less of them when we become richer (e.g. cheap dorm food). Recall from the previous section that it is possible for some or no goods to be inferior, but it is impossible for all goods to be inferior. 

Normally, the demand-curve for inferior goods is still downward-sloping; that is, we still have $\pdv{x_i}{p_i} < 0$. This is because even though we have positive income effect $-x_{i} \frac{\partial x_{i}}{\partial y}>0$, typically the negative substitution effect $\pdv{h_i}{p_i} < 0$ still dominates. Intuitively, an increase in the price of good $i$ has two effects for an inferior good: (1) it makes the consumer poorer, which influences consumption positively from the positive income effect of inferior goods, and (2) it makes the consumer substitute away to relatively cheaper options, which influences consumption negatively. If the second effect dominates, as is often the case, then the demand curve for an inferior good is still downward sloping.

However, it is theoretically possible for the former effect to dominate, causing demand to be \textit{upward-sloping}, or $\pdv{x_i}{p_i} > 0$. That is, when prices for good $i$ rise, the consumption of good $i$ rises. These goods are called \vocab{Giffen goods} and are a rare subset of inferior goods. This phenomenon occurs if 
$$
\bigg | x_{i} \frac{\partial x_{i}}{\partial y} \bigg | > \pdv{h_i}{p_i},
$$
which generally happens when 
$$x_{i} \frac{\partial x_{i}}{\partial y} \ll 0.$$
Intuitively, for the income effect to be so large, the good must be very inferior (i.e. $\varepsilon_y^i \ll 0$) and the budget share of $x_i$ must be very large, which is generally the case when the consumer is poor. A famous example of this phenomenon was presented by \href{https://www.ncbi.nlm.nih.gov/pmc/articles/PMC2964162/}{Jensen and Miller (2008)}. They found that subsidizing the price of rice in rural Hunan, China, resulted in decreased consumption of rice among poor households. Why? Poor households who used to spend a large fraction of their income on rice became effectively richer through this subsidy, so they no longer needed to buy as much rice since they could afford other foods in their diets.

\item[\vocab{Normal goods}] have positive income elasticity $\varepsilon_y^i > 0$, so their income effect is negative. That is, we consume more of them when we become richer; most goods we typically think of fall under this category. The demand curve is also unambiguously downward-sloping, since both the income effect and the substitution effect are negative. By the same reasoning as before, some or all of the goods in our option set must be normal.

It makes sense to buy more of a good when we become richer, but the relative size of this increase might be bigger for some goods than others. A normal good is a \vocab{luxury} if $\varepsilon_y^i > 1$, so a $1\%$ increase in income results in more than a $1\%$ increase in consumption of that good. The budget share of these goods increases as the consumer becomes richer. For example, a consumer might spend no money on luxury cars initially, but they might spend a sizable amount of money on luxury cars when they become rich.

Alternatively a normal good is a \vocab{necessity} if $\varepsilon_y^i < 1$, so a $1\%$ increase in income results in an increase in consumption of that good by between $0\%$ and $1\%$. The budget share of these goods decreases as the consumer becomes richer. For example, Engel's law is the observation that the budget share for food decreases when income rises, so food is a necessity.

In the special case where $\varepsilon_y^i = 1$ for all goods in the option set, we say that preferences are \vocab{homothetic}. That is, budget shares of all goods are fixed, so a $1\%$ increase in income causes a $1\%$ increase in the consumption of all goods.

\end{description}

\section{General Slutsky Equation}
In the original Slutsky equation, we were interested in the effect of a price change in good $i$ on the consumption of good $i$, so we differentiated with respect to $p_i$. If we were instead interested in the effect of a price change in some other good $j$ on the consumption of good $i$, we would differentiate with respect to $p_j$. The resulting derivation would look identical, and our final result would be the \vocab{general Slutsky equation}
$$
\boxed{\frac{\partial x_{i}}{\partial p_{j}}=\frac{\partial h_{i}}{\partial p_{j}}-x_{j} \frac{\partial x_{i}}{\partial y}.}
$$
The $-x_{j} \frac{\partial x_{i}}{\partial y}$ term represents the income effect, where the price of good $j$ makes the consumer poorer in real terms, which has some effect on the consumption of good $i$ depending on whether good $i$ is an inferior or normal good. However, whereas the substitution effect used to be unambiguously negative, the new substitution effect $\frac{\partial h_{i}}{\partial p_{j}}$ in the general equation can be positive or negative, depending on whether goods $i$ and $j$ are substitutes or complements, respectively.

\subsection*{Net and Gross Substitutes}
\TODO{Not sure if we want the separate definition and proposition environments in here}
Using this new substitution effect from the general Slutsky equation, we introduce two different ideas for how goods can be substitutes.

\begin{definition*}[Net substitutes]
We say that goods $i$ and $j$ are \vocab{net substitutes} if 
$$
\frac{\partial h_{i}}{\partial p_{j}}=\frac{\partial h_{j}}{\partial p_{i}}>0.
$$
\end{definition*}
Intuitively, two goods are net substitutes if an increase in the price of one good causes the consumption of the other good to rise, assuming that utility must be kept constant. We note that this definition is symmetric, which we prove in the following proposition.

\begin{proposition*}
Net substitutability is a symmetric condition. That is, $$\frac{\partial h_{i}}{\partial p_{j}}=\frac{\partial h_{j}}{\partial p_{i}}.$$ 
\end{proposition*}

\begin{proof}
By Shephard's Lemma, 
$$h_i(\vec{p}, \bar{u}) = \pdv{e(\vec{p}, \bar{u})}{p_i}.$$
Differentiating with respect to $p_j$ gives
$$\pdv{h_i(\vec{p}, \bar{u})}{p_j} = \frac{\partial^2 e(\vec{p}, \bar{u})}{\partial p_i \partial p_j}.$$
By Young's Theorem, we can swap the order of the partial derivatives, so 
$$\frac{\partial^2 e(\vec{p}, \bar{u})}{\partial p_i \partial p_j} = \frac{\partial^2 e(\vec{p}, \bar{u})}{\partial p_j \partial p_i}.$$
Thus, we conclude 
$$\pdv{h_i(\vec{p}, \bar{u})}{p_j} = \pdv{h_j(\vec{p}, \bar{u})}{p_i}$$
\end{proof}
We also claim that every good must have some net substitutes.
\begin{proposition*}
For all goods $j$, there exists good $i$ such that $\pdv{h_i}{p_j} > 0$.
\end{proposition*}
\begin{proof}
We know by the definition of the Hicksian that
$$u(\vec{h}(\vec{p}, \bar{u})) = \bar{u}.$$
Differentiating both sides by $p_j$ and using the multivariate chain rule yields 
$$
\sum_{i=1}^{n} u_{i} \frac{\partial h_{i}}{\partial p_{j}}=0.
$$
By the assumption of non-satiation (i.e. marginal utility for every good is always positive), $u_i > 0$ for all $i$. By the law of compensated demand, which we showed previously in this chapter, we have $\pdv{h_j}{p_j} < 0$. This means there must be some good $i$ such that $\pdv{h_i}{p_j} > 0$.
\end{proof}

If we want to examine possible substitution when budget is held fixed rather than when utility is held fixed, we have a similar definition using Marshallian demand instead of Hicksian demand.

\begin{definition*}[Gross substitutes]
We say that good $i$ is a \vocab{gross substitute} for good $j$ if
$$\pdv{x_i}{p_j} > 0.$$
\end{definition*}
That is, good $i$ is a gross substitute for good $j$ if an increase in the price of $j$ results in increased consumption of $i$ under a fixed budget. Notice that this property is not symmetric. For a counterexample, consider two goods $x$ and $y$ with a quasilinear utility function $U(a, b) = a + \log b$. We encourage this counterexample as an exercise to the reader, but in this case, $b$ is a gross substitute for $a$, but not vice versa.
\TODO{I'm not sure if this is mean}

\section{Slutsky Equation with Two Goods}

If we have only two goods, we can derive Slutsky's equation directly for more intuition. We start with the utility-maximization problem 
$$
\max _{a, b \geq 0} u(a, b) \text { s.t. } p_{a} a+p_{b} b \leq y.
$$
We substitute the budget constraint to obtain the unconstrained univariate optimization problem 
$$
\max _{a \in\left[0, y / p_{a}\right]} u\left(a, \frac{y-p_{a} a}{p_{b}}\right).
$$
The first order condition is 
$$
u_{a}\left(a^{*}, b^{*}\right)-\frac{p_{a}}{p_{b}} u_{b}\left(a^{*}, b^{*}\right)=0.
$$
Together with the budget constraint 
$$
b^{*}=\frac{y-p_{a} a^{*}}{p_{b}},
$$
these two conditions give implicit solutions for Marshallian demands $a^*(p_a, p_b, y)$ and $b^*(p_a, p_b, y)$.
We differentiate the first order condition with respect to $p_a$ with the Implicit Function Theorem, yielding 
$$
-\frac{u_{b}}{p_{b}}+\left(u_{a a}-\frac{p_{a}}{p_{b}} u_{a b}\right) \frac{\partial a^{*}}{\partial p_{a}}+\left(u_{a b}-\frac{p_{a}}{p_{b}} u_{b b}\right) \frac{\partial b^{*}}{\partial p_{a}}=0.
$$
Substituting our expression for $b^*$ from the budget constraint yields 
$$
-\frac{u_{b}}{p_{b}}+\left(u_{a a}-\frac{p_{a}}{p_{b}} u_{a b}\right) \frac{\partial a^{*}}{\partial p_{a}}-\left(u_{a b}-\frac{p_{a}}{p_{b}} u_{b b}\right) \frac{1}{p_{b}}\left(a^{*}+p_{a} \frac{\partial a^{*}}{\partial p_{a}}\right)=0.
$$
We can rearrange these terms to get
$$
-\left[u_{a a}-2 \frac{p_{a}}{p_{b}} u_{a b}+\left(\frac{p_{a}}{p_{b}}\right)^{2} u_{b b}\right] \frac{\partial a^{*}}{\partial p_{a}}=-\frac{u_{b}}{p_{b}}-\frac{1}{p_{b}}\left(u_{a b}-\frac{p_{a}}{p_{b}} u_{b b}\right) a^{*}.
$$

What does all of this mean? The left hand side is the comparative static $\pdv{a^*}{p_a}$, multiplied by
$$
S \equiv -\left[u_{a a}-2 \frac{p_{a}}{p_{b}} u_{a b}+\left(\frac{p_{a}}{p_{b}}\right)^{2} u_{b b}\right],
$$
which is the value that must be positive in the second-order condition for a maximum. We see the substitution effect 
$$
-\frac{u_{b}}{S p_{b}}<0,
$$
which we can verify is the same as $\pdv{h_a}{p_a}$ if we set up the expenditure-minimization problem accordingly. Lastly, we see the income effect 
$$
-\frac{1}{S p_{b}}\left(u_{a b}-\frac{p_{a}}{p_{b}} u_{b b}\right) a^{*}
$$
which is equal to $-\frac{\partial a^{*}}{\partial y} a^{*}$ if we compute $\frac{\partial a^{*}}{\partial y}$ with the implicit function theorem.

\section*{Recap}